
\documentclass[../AdvancementSummary.tex]{subfiles}
\begin{document}

%%%%%%%%%%%%%%%%%%%%%%%%%%%%%%%%%%%%%%%%%%%%%%%%%%%
\section{Abstract}
%%%%%%%%%%%%%%%%%%%%%%%%%%%%%%%%%%%%%%%%%%%%%%%%%%%

Disordered proteins are present in at least 40\% of human proteins, including signaling molecules. What and how disordered proteins contribute to a signaling network is not well understood. \hl{I don't like this sentence} Here we explore how disordered proteins impact singular and multi-site ligand binding through a variety of disorder-specific phenomenons. In addition to basic properties of disordered protein interactions, we investigate disorder-to-order transitions, electrostatic membrane association, simultaneous ligand binding and effects of surface presence. We find that disordered proteins may create positive or negative cooperativity and intrinsic sequential binding.  These effects are influenced by the length of the disordered protein, size of the binding ligand, location of the binding sites along the polymer and presence or absence of a surface. Intrinsically disordered proteins themselves may therefore act as signaling modules that contribute complex signaling behavior to a network.



\end{document}


