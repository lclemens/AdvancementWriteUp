%\documentclass[aps,pre,twocolumn]{revtex4}
%\documentclass[aps,pre,onecolumn]{revtex4}
\documentclass[onecolumn]{article}
\usepackage[numbers]{natbib}
\bibliographystyle{plainnat}
\usepackage[margin=1in]{geometry}
\usepackage[framed]{mcode}


%\bibliographystyle{prsty}

\usepackage{epsfig}
\usepackage{amsmath}
\usepackage{color,soul,subcaption,float}
\usepackage{listings}
\input preamble.tex

\usepackage{multirow}

\setlength{\parskip}{10pt plus 1pt minus 1pt}
\setlength{\parindent}{0pt}

\begin{document}
\newcommand{\citel}[1]{\citep{#1}}
\newcommand{\citelt}[1]{\citet{#1}}

\def\Tevac{T_{1/2}^{\mbox{\,\scriptsize evac}}}
\def\Treturn{T_{1/2}^{\mbox{\,\scriptsize return}}}
\def\toffbar{\bar{t}_{\mbox{\scriptsize off}}}
\def\toff{{t}_{\mbox{\scriptsize off}}}
\def\Peclet{P\'{e}clet}


%% Make title
\begin{center}
\Large
Signaling Behaviors of Intrinsically Disordered Proteins

\large
Lara Clemens
\date{}

\end{center}
\normalsize

%%%%%%%%%%%%%%%%%%%%%%%%%%%%%%%%%%%%%%%%%%%%%%%%%%%
%\section{Executive Summary}
%%%%%%%%%%%%%%%%%%%%%%%%%%%%%%%%%%%%%%%%%%%%%%%%%%%

% Disordered proteins exist
Disordered domains exist in at least 40\% of human proteins. 
% They serve many purposes
They take on many functions, including roles in tethering, signal propagation, and actin polymerization. 
% How these functions are regulated is not well understood, so model
It is unknown if tuning characteristics of the disordered domain (e.g. length, binding site locations) will impact these functions, and whether this is a mechanism of regulation. 
% Previous models
Previous models of disordered proteins have shown a freely-jointed chain (FJC) to match experimental results. 
% here we
\textbf{Here we extend the FJC model, simulating polymers and ligands using computational tools from polymer physics. }
% look it works
Modeling disordered proteins in this manner has already offered explanatory power. \textbf{In our previous work, we implement this model to demonstrate a method by which the disordered protein formin can accelerate actin polymerization} via G-actin delivery to the growing end, despite forces extending the polymer [Bryant et al. Cytoskeleton 2017]. 

% TCR
Of particular interest, disordered signaling proteins can exhibit complex signaling behavior such as cooperative binding rate enhancement. For example, a multiply-phosphorylated subunit of a T cell receptor, the CD3$\zeta$ chain, experiences a 100-fold increase in the binding rate of a kinase to its tyrosines from the first to the sixth phosphorylation [Mukhopadhyay et al. Biophys J 2016]. How disordered domains impart nonlinear signaling mechanisms to a network is not well understood. \textbf{In our previous work, we use a FJC model of the CD3$\zeta$ chain to show a 100-fold change in the phosphorylation rate is achievable} through a disordered-to-ordered transition upon phosphorylation [Mukhopadhyay Biophys J 2016]. In the following work, \textbf{we extend this model to investigate how intrinsic characteristics of disorder may culminate in complex signaling}, particularly in the case of the T Cell Receptor CD3$\zeta$ chain.

%We model disordered proteins to investigate how intrinsic characteristics of disorder may culminate in complex signaling, particularly in the case of the T Cell Receptor CD3$\zeta$ chain.



%
%% Disordered proteins exist
%Disordered domains exist in at least 40\% of human proteins. 
%% They serve many purposes
%They take on many functions, including roles in tethering, signal propagation, and actin polymerization. 
%% How these functions are regulated is not well understood, so model
%Within signaling pathways, disordered subunits can exhibit complex signaling behavior such as cooperative binding rate enhancement. For example, a multiply-phosphorylated subunit of a T cell receptor, the CD3$\zeta$ chain, experiences a 100-fold increase in the binding rate of a kinase to its tyrosines from the first to the sixth phosphorylation. [Mukhopadhyay et al. Biophys J 2016] How disordered domains impart nonlinear signaling mechanisms to a network is not well understood. We model disordered proteins to investigate how intrinsic characteristics of disorder may culminate in complex signaling, particularly in the case of the T Cell Receptor CD3$\zeta$ chain. In our previous work, we show that 
%
%Previous models of disordered proteins have shown a freely-jointed chain (FJC) to match experimental results. We generalize and extend the FJC model in our studies, implementing a Monte Carlo Metropolis algorithm to calculate pseudo-equilibrium statistics for simulated polymers and ligands. Modeling disordered proteins in this manner has already offered explanatory power.. \textbf{In our previous work, we implement this model to demonstrate a method by which the disordered protein formin can accelerate actin polymerization via G-actin delivery to the growing end, despite forces extending the polymer. [Bryant et al. Cytoskeleton 2017] 

\subsection*{Project 1: Steric Occlusion, Disorder-to-Order Transitions}

We investigate how properties of disordered proteins influence ligand binding. Properties we explore are length, ligand size, location of binding site and membrane-anchored versus cytosolic. We find that longer polymers and larger ligands each decrease the binding rate of a ligand. The binding site location is significantly more available when at the end of the polymer than in the middle due to the middle regions experiencing more conformations where the polymer is bundled around the binding site. Membrane-anchored disordered proteins are stretched out slightly, exposing the binding sites more often. However, this effect is outweighed by the membrane occluding the ligand when the binding site is oriented too close to the membrane. This creates an overall decrease in the binding availability of site in half-space compared to free-space.

Disorder-to-order transitions occur as a result of post-translational modifications in numerous proteins. \textbf{When we include local disorder-to-order transitions as a consequence of phosphorylation, we see an increase in the average binding rate of ligands to subsequent tyrosines}, with the magnitude dependent on the degree of local structuring. When we assume dephosphorylation to be an unstructuring event, we find a reduction of dephosphorylation rate as more sites are dephosphorylated. If we permit the same, moderate amount of local structuring per phosphorylation as local unstructuring per dephosphorylation, then we have a greater increase in binding rates per phosphorylation than decrease per dephosphorylation. \textbf{In the future, we will investigate how reversible phosphorylation impacts the nonlinear signaling behavior of the system.}

\subsection*{Project 2: Electrostatic Membrane Association}

Current experimental studies of the TCR CD3$\zeta$ chain indicate it is associated with the lipid bilayer in untriggered TCRs. This association is mitigated by mutation of the basic residues of the chain, phosphorylation of the tyrosines, or by calcium influx. It has been shown that immediately after TCR triggering, a calcium influx will occur, allowing more $\zeta$-chains to dissociate. However, since TCR triggering is the causal event of T cell signaling, it is unclear how the first $\zeta$-chains dissociate from the membrane to propagate the signal. We wish to investigate how a single chain may become phosphorylated while primarily associated with the membrane. \textbf{In this project, we have implemented an electrostatic potential to simulate the attraction of basic residues to the membrane. Future work will include tuning parameters on our model of membrane-associated CD3$\zeta$ to explore the accessibility of tyrosines to a kinase and whether cooperative effects can arise from phosphorylation} of the membrane-associated chain. These cooperative effects may help explain how initial TCR signaling occurs.

\subsection*{Project 3: Simultaneous Binding}

Multiple ligands have been shown to simultaneously attach to a receptor. It is not clear how one bound ligand will influence the accessibility of the receptor for subsequent ligands. In this project, we simulate the effect of multiple bound ligands on the disordered protein. \textbf{We find that having simultaneously bound ligands on a disordered polymer will create a negative cooperative effect}, making each successive tyrosine harder for a ligand to bind. This effect increases with ligand radius, where large ligands can create a 30-fold decrease of the binding rate from the first to the sixth bound ligand.

A more complete model of simultaneous binding will include all of the chains of a receptor. \textbf{In the future, we will model each of the six disordered tails of the T Cell Receptor and find estimates for how many molecules bind to a receptor on average.} These results may be compared with experimental studies for ZAP-70 binding to the cytoplasmic regions of the T Cell Receptor.

\subsection*{Project 4: Surface Effects on Local Concentration}

Disordered regions often appear in clustered reactions, acting as tethers to a surface or between domains, e.g. reactions tethered to the membrane. Experimental studies of such reactions are often easier to execute on a matrix, such as dextran, instead of on a surface. However, the presence of a surface has been shown to influence the characteristics of a disordered protein, so it is natural to think it may influence the effective concentration two disordered proteins may see of each other. In this project, we simulate two disordered domains, one attached with a ligand, in free space and half space to determine how tethering to a matrix versus a surface impacts the binding of a tethered ligand to a tethered binding site. \textbf{Initial results indicate that presence of a surface does not influence the reach parameter of the disordered proteins but does influence the effective concentration.} \textbf{In the future, we will determine more precisely how reactions performed on a matrix are related to those on a surface} and if there exists a simple conversion factor or function to describe the relationship. This might suggest if the membrane plays a regulatory role in membrane-bound reactions.






%%%%%%%%%%%%%%%%%%%%%%%%%%%%%%%%%%%%%%%%%%%%%%%%%%%
\end{document}
%%%%%%%%%%%%%%%%%%%%%%%%%%%%%%%%%%%%%%%%%%%%%%%%%%%





