%\documentclass[aps,pre,twocolumn]{revtex4}
%\documentclass[aps,pre,onecolumn]{revtex4}
\documentclass[onecolumn]{article}
\usepackage[numbers]{natbib}
\bibliographystyle{plainnat}
\usepackage[margin=0.5in]{geometry}
\usepackage[framed]{mcode}


%\bibliographystyle{prsty}

\usepackage{epsfig}
\usepackage{amsmath}
\usepackage{color,soul,subcaption,float}
\usepackage{listings}
\input preamble.tex

\usepackage{multirow}

\setlength{\parskip}{10pt plus 1pt minus 1pt}
\setlength{\parindent}{0pt}

\begin{document}
\newcommand{\citel}[1]{\citep{#1}}
\newcommand{\citelt}[1]{\citet{#1}}

\def\Tevac{T_{1/2}^{\mbox{\,\scriptsize evac}}}
\def\Treturn{T_{1/2}^{\mbox{\,\scriptsize return}}}
\def\toffbar{\bar{t}_{\mbox{\scriptsize off}}}
\def\toff{{t}_{\mbox{\scriptsize off}}}
\def\Peclet{P\'{e}clet}


%% Make title
\begin{center}
\Large
Disordered proteins create nonlinear signaling behavior

\large
Lara Clemens
\date{}

\end{center}
\normalsize

%%%%%%%%%%%%%%%%%%%%%%%%%%%%%%%%%%%%%%%%%%%%%%%%%%%
%\section{Executive Summary}
%%%%%%%%%%%%%%%%%%%%%%%%%%%%%%%%%%%%%%%%%%%%%%%%%%%


Disordered domains exist in at least 40\% of human proteins. They take on many functions, including roles in tethering, signal propagation, and actin polymerization. Within signaling pathways, disordered subunits can exhibit complex signaling behavior such as cooperative binding rate enhancement. For example, a multiply-phosphorylated subunit of a T cell receptor, the CD3$\zeta$ chain, experiences a 100-fold increase in the binding rate of a kinase to its tyrosines from the first to the sixth phosphorylation. [Mukhopadhyay et al. Biophys J 2016] How disordered domains impart nonlinear signaling mechanisms to a network is not well understood. We model disordered proteins to investigate how intrinsic characteristics of disorder may culminate in complex signaling, particularly in the case of the T Cell Receptor CD3$\zeta$ chain. 

Previous models of disordered proteins have shown a freely-jointed chain (FJC) to match experimental results. Our previous work on the disordered protein formin demonstrates the explanatory power of this type of model. Modeling formin as a FJC, we propose a mechanism for formin to accelerate actin polymerization via G-actin delivery to the growing end, despite forces extending the polymer. [Bryant et al. Cytoskeleton 2017] We extend this model in our studies, implementing a Monte Carlo Metropolis algorithm to calculate pseudo-equilibrium statistics for the simulated polymer and ligands. 

\subsection*{Project 1: Steric Occlusion, Disorder-to-Order Transitions and Electrostatics}

We investigate how properties of disordered proteins influence ligand binding. Properties we explore are length, ligand size, location of binding site and membrane-anchored versus cytosolic. We find that longer polymers and larger ligands each decrease the binding rate of a ligand. The binding site location is significantly more available when at the end of the polymer than in the middle due to the middle regions experiencing more conformations where the polymer is bundled around the binding site. Membrane-anchored disordered proteins are stretched out slightly, exposing the binding sites more often. However, this effect is outweighed by the membrane occluding the ligand when the binding site is oriented too close to the membrane. This creates an overall decrease in the binding availability of site in thalf-space compared to free-space.

Disorder-to-order transitions occur as a result of post-translational modifications in numerous proteins. When we include this phenomenon as a consequence of phosphorylation events, we see an increase in the average binding rate of ligands to subsequent tyrosines. This transition creates a cooperative effect which increases the rate of binding from 3-10 fold dependent on the degree of local structuring. If we allow dephosphorylation to be an unstructuring event, then we introduce negative cooperativity to the system. If we permit the same, moderate amount of local structuring per phosphorylation as local unstructuring per dephosphorylation, then we have a greater increase in binding rates per phosphorylation than decrease per dephosphorylation.

Current experimental studies of the TCR CD3$\zeta$ chain indicate it is associated with the lipid bilayer. This association is mitigated by either mutation of the basic residues of the chain or by phosphorylation of the tyrosines. We wish to investigate how a single chain may become phosphorylated while primarily associated with the membrane. Future work will include tuning parameters on our model of membrane-associated CD3$\zeta$ to remark on how accessible a single tyrosine is to a kinase and whether cooperative effects can arise from phosphorylation of the membrane-associated chain.

\subsection*{Project 2: Simultaneous Binding}

Multiple ligands have been shown to simultaneously attach to a receptor. A bound ligand will help straighten a disordered chain, increasing accessibility of binding sites, but will also sterically prevent other ligands from binding. Therefore, it is not obvious how a bound ligand will influence the accessibility of a binding site. We find that having simultaneously bound ligands will create a negative cooperative effect, making each successive tyrosine harder for a ligand to bind. This effect scales with ligand radius, where large ligands can create a 30-fold decrease of the binding rate from the first to the sixth phosphorylation.

A more complete model of simultaneous binding will include all of the chains of a receptor. In the case of CD3$\zeta$, we will model each of the six disordered tails and find estimates for how many molecules bind to a receptor on average. These results may be compared with experimental studies for ZAP-70 binding to the cytoplasmic regions of the T Cell Receptor.


\subsection*{Project 3: Surface Effects on Local Concentration}

Disordered regions often appear in clustered reactions, acting as tethers to a surface or between domains. Experimental studies of such reactions are often easier to execute on a matrix, such as dextran, instead of a surface. However, the presence of a surface has been shown to influence the characteristics of a disordered protein, so it is natural to think it may influence the effective concentration two disordered proteins may see of each other. We simulate two disordered domains, one attached with a ligand, to determine how tethering to a surface versus a matrix impacts the binding of a tethered ligand to a tethered binding site. Initial results indicate that the surface does not influence the reach parameter of the disordered proteins but does influence the effective concentration. Further study will determine more precisely how we may relate studies performed on a matrix to those on a surface and if there exists a simple conversion factor or function to describe the relationship.






%%%%%%%%%%%%%%%%%%%%%%%%%%%%%%%%%%%%%%%%%%%%%%%%%%%
\end{document}
%%%%%%%%%%%%%%%%%%%%%%%%%%%%%%%%%%%%%%%%%%%%%%%%%%%





